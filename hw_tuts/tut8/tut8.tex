\documentclass{article}
\usepackage{geometry,amsmath,amssymb,theorem,caption,extarrows,mathrsfs,physics,bm}
\usepackage{graphicx,xcolor,listings,geometry,booktabs,subfigure,tikz}
\usepackage{pgfplots,grffile}
\pgfplotsset{compat=newest}
  %% the following commands are needed for some matlab2tikz features
\usetikzlibrary{plotmarks}
\usetikzlibrary{arrows.meta}
\usetikzlibrary{calc}
\usepgfplotslibrary{patchplots}
\usepackage{xeCJK,fontspec}
\setCJKmainfont[BoldFont=方正小标宋简体,ItalicFont=楷体]{华文宋体}
  \setCJKsansfont{黑体}%serif是有衬线字体sans serif无衬线字体。
  %\setmonofont{CMU Typewriter Text} % 等寬字型
  \XeTeXlinebreaklocale "zh"
  \XeTeXlinebreakskip = 0pt plus 1pt minus 0.1pt
\lstset{
    basicstyle          =   \sffamily,          % 基本代码风格
    keywordstyle        =   \bfseries,          % 关键字风格
    commentstyle        =   \rmfamily\itshape,  % 注释的风格,斜体
    stringstyle         =   \ttfamily,  % 字符串风格
    flexiblecolumns,                % 别问为什么,加上这个
    numbers             =   left,   % 行号的位置在左边
    showspaces          =   false,  % 是否显示空格,显示了有点乱,所以不现实了
    numberstyle         =   \small\ttfamily,    % 行号的样式,小五号,tt等宽字体
    showstringspaces    =   false,
    captionpos          =   t,      % 这段代码的名字所呈现的位置,t指的是top上面
    frame               =   tb,   % 显示边框
}

\lstdefinestyle{Python}{
    language        =   Python, % 语言选Python
    basicstyle      =   \small\ttfamily,
    numberstyle     =   \small\ttfamily,
    keywordstyle    =   \color{blue},
    keywordstyle    =   [2] \color{teal},
    stringstyle     =   \color{magenta},
    commentstyle    =   \color[HTML]{338AAF}\ttfamily,
    breaklines      =   true,   % 自动换行,建议不要写太长的行
    columns         =   fixed,  % 如果不加这一句,字间距就不固定,很丑,必须加
    basewidth       =   0.5em,
}
\definecolor{codegreen}{rgb}{0,0.6,0}
\definecolor{codegray}{rgb}{0.5,0.5,0.5}
\definecolor{codepurple}{rgb}{0.58,0,0.82}
\definecolor{backcolour}{rgb}{0.95,0.95,0.92}
\lstdefinestyle{mathematica}{
    backgroundcolor=\color{backcolour},
    commentstyle=\color[HTML]{338AAF}\ttfamily,
    keywordstyle=\small\sffamily\bfseries\color{magenta},
    numberstyle=\small\color{codegray},
    stringstyle=\color{codepurple},
    basicstyle=\small\ttfamily,
    breakatwhitespace=false,
    breaklines=true,
    basewidth=0.5em,
    captionpos=b,
    columns=fixed,
    keepspaces=true,
    numbers=left,
    numbersep=5pt,
    showspaces=false,
    showstringspaces=false,
    showtabs=false,
    tabsize=4
}
\lstdefinestyle{matlab}{
    language=matlab,
    backgroundcolor=\color{backcolour},
    commentstyle=\color[HTML]{338AAF}\ttfamily,
    keywordstyle=\small\sffamily\bfseries\color{magenta},
    numberstyle=\small\color{codegray},
    stringstyle=\color{codepurple},
    basicstyle=\small\ttfamily,
    breakatwhitespace=false,
    breaklines=true,
    basewidth=0.5em,
    captionpos=b,
    columns=fixed,
    keepspaces=true,
    numbers=left,
    numbersep=5pt,
    showspaces=false,
    showstringspaces=false,
    showtabs=false,
    tabsize=4
}
\newcommand{\dif}{\mathop{}\!\mathrm{d}}
\newcommand{\const}{\mathop{}\!\mathrm{const.}}
\newcommand{\splitline}{\noindent\rule[0.25\baselineskip]{\textwidth}{0.5pt}}
\newcommand{\autographinsert}[2]{\includegraphics[
  height=\dimexpr\pagegoal-\pagetotal-4\baselineskip\relax,width=#1\textwidth,
  keepaspectratio]{#2}}
  % NOTE: 插入图片如果出问题飘到下一页,请调整减掉的行数
\newtheorem{theorem}{定理}
\geometry{a4paper,left=1.8cm,right=1.8cm,top=1.5cm,bottom=1.5cm}
\begin{document}
\title{Tutorial 8}
\author{仇琨元}
\date{\today}
\maketitle

\section{Problem 1}

\paragraph{(a)}

The sequence \( y[n] \) is upsampled from \( x[n] \) with upsampling rate L. Use the DFT spectrum of \( x[n] \) to express \( Y[k] \):
\begin{equation}
	\begin{aligned}
		Y[k] & =\sum_{n=0}^{LN-1} y[n]W_{LN}^{nk}  \\
		     & =\sum_{n=0}^{N-1} x[n] W_{LN}^{nkL} \\
	\end{aligned}
\end{equation}

Use the unitary characteristics of the twiddling factor,
\begin{equation}
	(W_{N}^{n})^{QN+R}=W_{N}^{Rn}\exp(-2\pi jQn)=W_{N}^{Rn}
\end{equation}

Where
\begin{equation}
	Q=k//N\in \mathbb{N},R=k\mod N\in \mathbb{N}
\end{equation}

Substitute (3) into (1), thus
\begin{equation}
	Y[k]=\text{repmat}(X[k],L)
\end{equation}

\paragraph{(b)}

Use MATLAB to plot the result:
\begin{figure}[htpb]
	\centering
	% This file was created by matlab2tikz.
%
%The latest updates can be retrieved from
%  http://www.mathworks.com/matlabcentral/fileexchange/22022-matlab2tikz-matlab2tikz
%where you can also make suggestions and rate matlab2tikz.
%
\definecolor{mycolor1}{rgb}{0.00000,0.44700,0.74100}%
%
\begin{tikzpicture}

\begin{axis}[%
width=4.521in,
height=3.566in,
at={(0.758in,0.481in)},
scale only axis,
xmin=0,
xmax=20,
ymin=0,
ymax=4,
axis background/.style={fill=white}
]
\addplot[ycomb, color=mycolor1, mark=*, mark options={solid, mycolor1}, forget plot] table[row sep=crcr] {%
1	1\\
2	1\\
3	4\\
4	3\\
5	2\\
6	1\\
7	1\\
8	4\\
9	3\\
10	2\\
11	1\\
12	1\\
13	4\\
14	3\\
15	2\\
16	1\\
17	1\\
18	4\\
19	3\\
20	2\\
};
\addplot[forget plot, color=white!15!black] table[row sep=crcr] {%
0	0\\
20	0\\
};
\end{axis}
\end{tikzpicture}%
	\label{fig:p1}
\end{figure}

\section{Problem 2}
\paragraph{(a)}

\begin{lstlisting}[language=matlab,style=matlab]
    x=[2,-5,6,-3,4,-4,0,-7,8];
    y=circshift(x,-12);
    y(-3-(-5)+1)
\end{lstlisting}

The answer is -4.

\paragraph{(b)}

\begin{lstlisting}[language=matlab,style=matlab]
    z=circshift(x,15);
    z(2-(-5)+1)
\end{lstlisting}

The answer is -5.

\section{Problem 3}

Use the convolution characteristics of DFT to evaluate the discrete circular convolution.

\begin{lstlisting}[language=matlab,style=matlab]
    ifft(fft([-3,2,-1,4]).*fft([1,3,2,-2]))
\end{lstlisting}

Therefore
\begin{equation}
	y[n]=[3,3,-9,11]
\end{equation}

\section{Problem 4}
\paragraph{(a)}

The circular convolution for two sequences with the same length can be expressed as the convolution between two extended sequence:
\begin{equation}
	\begin{aligned}
		y_{C}[n] & =(g[n]*(\delta[n]+\delta[n-L])*h[n])[L,2L-1] \\
		         & =(y_{L}[n]*(\delta[n]+\delta[n-L]))[L,2L-1]  \\
		         & =(y_{L}[n])[L,2L-1]+(y_{L}[n])[0,L-1]
	\end{aligned}
\end{equation}

Where the \( L \) is the length of the operand and there should be zero-padding in the lase summation:
\begin{lstlisting}[language=matlab,style=matlab]
    yl=conv(g,h);
    l=length(yl);
    yc=[yl(l+1:2*l),0]+yl(1:l+1)
\end{lstlisting}

\paragraph{(b)}

Use the formula derived above, the result of the circular equation is:
\begin{equation}
	\begin{aligned}
		y_{L}             & =[-6,22,-3,-54,77,9,-28,63,-6,13,12] \\
		\Rightarrow y_{C} & =[-34,85,-9,-41,89,9]
	\end{aligned}
\end{equation}

This result corresponds to that calculated by IFFT.

\section{Problem 5}

Firstly, use the convolution, TD shifting and decimation property of DFT to simplify the \( G[k],H[k] \):
\begin{equation}
	\begin{aligned}
		G[k]             & =\frac{1}{2}\left( X[2k]+W_{N}^{k}X[2k] \right),k\in[0,\frac{N}{2}-1] \\
		H[k]             & =\frac{1}{2}\left( X[2k]-W_{N}^{k}X[2k] \right),k\in[0,\frac{N}{2}-1] \\
		\Rightarrow X[k] & =\frac{1}{2}(G[k]^{-1}+H[k]^{-1})                                     \\
		                 & =(1+W_{N}^{m})G[m]+(1-W_{N}^{m})H[m],m=k \mod \frac{N}{2}
	\end{aligned}
\end{equation}

\section{Problem 6}
\paragraph{(a)}

For arbitrary length \( N \), the asymmetric sequence \( x[n] \) can be written as
\begin{equation}
	x[n]=x[0:\left\lfloor \frac{N}{2} \right\rfloor ]+x[0:\left\lfloor \frac{N}{2} \right\rfloor+1]+x_{c}
\end{equation}

where the \( x_{c} \) can be zero for even N and non-zero for odd N.

\section{Problem 7}

Use DFT to evaluate the linear convolution. Pad the input vectors with zeros that the length of both padded vectors exceed \( L_{1}+L_{2}-1 \):
\begin{lstlisting}[language=matlab,style=matlab]
    x=[2,1,2];
    y=[-4,0,-3,2];
    x1=[x,zeros([1,length(y)-1])];
    y1=[y,zeros([1,length(x)-1])];
    c=ifft(fft(x1).*fft(y1));
\end{lstlisting}

The result is
\begin{equation}
	w[n]=[-8,-4,-14,1,-4,4]
\end{equation}


\end{document}